%% 
%% Copyright 2007, 2008, 2009 Elsevier Ltd
%% 
%% This file is part of the 'Elsarticle Bundle'.
%% ---------------------------------------------
%% 
%% It may be distributed under the conditions of the LaTeX Project Public
%% License, either version 1.2 of this license or (at your option) any
%% later version.  The latest version of this license is in
%%    http://www.latex-project.org/lppl.txt
%% and version 1.2 or later is part of all distributions of LaTeX
%% version 1999/12/01 or later.
%% 
%% The list of all files belonging to the 'Elsarticle Bundle' is
%% given in the file `manifest.txt'.
%% 
\documentclass[a4paper,12pt]{article}

\usepackage[xcolor,qtwo]{rvdtx}
\usepackage{multicol}
\usepackage{color}
\usepackage{xspace}
\usepackage{pdfwidgets}
\usepackage{enum}

\def\ttdefault{cmtt}

\headsep4pc

\makeatletter
\def\bs{\expandafter\@gobble\string\\}
\def\lb{\expandafter\@gobble\string\{}
\def\rb{\expandafter\@gobble\string\}}
\def\@pdfauthor{C.V.Radhakrishnan}
\def\@pdftitle{elsarticle.cls -- A documentation}
\def\@pdfsubject{Document formatting with elsarticle.cls}
\def\@pdfkeywords{LaTeX, Elsevier Ltd, document class}
\def\file#1{\textsf{#1}\xspace}

%\def\LastPage{19}

\DeclareRobustCommand{\LaTeX}{L\kern-.26em%
        {\sbox\z@ T%
         \vbox to\ht\z@{\hbox{\check@mathfonts
           \fontsize\sf@size\z@
           \math@fontsfalse\selectfont
          A\,}%
         \vss}%
        }%
     \kern-.15em%
    \TeX}
\makeatother

\def\figurename{Clip}

\begin{document}

\def\testa{This is a specimen document. }
\def\testc{\testa\testa\testa\testa}
\def\testb{\testc\testc\testc\testc\testc}
\long\def\test{\testb\par\testb\par\testb\par}

\pinclude{\copy\contbox\printSq{\LastPage}}

\title{elsarticle.cls -- A better way to format your document}

\author{Elsevier Ltd}
\contact{elsarticle@river-valley.com}

\version{1.20}
\date{\today}
\maketitle

\section{Introduction}
%\hypertarget{introduction}{}

\file{elsarticle.cls} is a thoroughly re-written document class
for formatting \LaTeX{} submissions to Elsevier journals.
The class uses the environments and commands defined in \LaTeX{} kernel
without any change in the signature so that clashes with other
contributed \LaTeX{} packages such as \file{hyperref.sty},
\file{preview-latex.sty}, etc., will be minimal.
\file{elsarticle.cls} is primarily built upon the default
\file{article.cls}.  This class depends on the following packages
for its proper functioning:

\begin{enumerate}
\item \file{pifont.sty} for openstar in the title footnotes;
\item \file{natbib.sty} for citation processing;
\item \file{geometry.sty} for margin settings;
\item \file{fleqn.clo} for left aligned equations;
\item \file{graphicx.sty} for graphics inclusion;
\item \file{txfonts.sty} optional font package, if the document is to
  be formatted with Times and compatible math fonts;
\item \file{hyperref.sty} optional packages if hyperlinking is
  required in the document.

\end{enumerate}

All the above packages are part of any standard \LaTeX{} installation.
Therefore, the users need not be bothered about downloading any
extra packages.  Furthermore, users are free to make use of \textsc{ams}
math packages such as \file{amsmath.sty}, \file{amsthm.sty},
\file{amssymb.sty}, \file{amsfonts.sty}, etc., if they want to.  All
these packages work in tandem with \file{elsarticle.cls} without
any problems.

\section{Major Differences}
%\hypertarget{majordifferences}{}

Following are the major differences between \file{elsarticle.cls}
and its predecessor package, \file{elsart.cls}:

\begin{enumerate}[\textbullet]
\item \file{elsarticle.cls} is built upon \file{article.cls}
while \file{elsart.cls} is not. \file{elsart.cls} redefines
many of the commands in the \LaTeX{} classes/kernel, which can
possibly cause surprising clashes with other contributed
\LaTeX{} packages;

\item provides preprint document formatting by default, and
optionally formats the document as per the final
style of models $1+$, $3+$ and $5+$ of Elsevier journals;

\item some easier ways for formatting \verb+list+ and
\verb+theorem+ environments are provided while people can still
use \file{amsthm.sty} package;

\item \file{natbib.sty} is the main citation processing package
  which can comprehensively handle all kinds of citations and
works perfectly with \file{hyperref.sty} in combination with
\file{hypernat.sty};

\item long title pages are processed correctly in preprint and
  final formats.

\end{enumerate}

\section{Installation}
%\hypertarget{installation}{}

The package is available at author resources page at Elsevier
(\url{http://www.elsevier.com/locate/latex}).
It can also be found in any of the nodes of the Comprehensive
\TeX{} Archive Network (\textsc{ctan}), one of the primary nodes
being
\url{http://www.ctan.org/tex-archive/macros/latex/contrib/elsevier/}.
Please download the \file{elsarticle.dtx} which is a composite
class with documentation and \file{elsarticle.ins} which is the
\LaTeX{} installer file. When we compile the
\file{elsarticle.ins} with \LaTeX{} it provides the class file,
\file{elsarticle.cls} by
stripping off all the documentation from the \verb+*.dtx+ file.
The class may be moved or copied to a place, usually,
\verb+$TEXMF/tex/latex/elsevier/+, %$%%%%%%%%%%%%%%%%%%%%%%%%%%%%
or a folder which will be read                   
by \LaTeX{} during document compilation.  The \TeX{} file
database needs updation after moving/copying class file.  Usually,
we use commands like \verb+mktexlsr+ or \verb+texhash+ depending
upon the distribution and operating system.


\section{Usage}\label{sec:usage}
%\hypertarget{usage}{}
The class should be loaded with the command:

\begin{vquote}
 \documentclass[<options>]{elsarticle}
\end{vquote}

\noindent where the \verb+options+ can be the following:

\begin{description}

\item [{\tt\color{verbcolor} preprint}]  default option which format the
  document for submission to Elsevier journals.

\item [{\tt\color{verbcolor} review}]  similar to the \verb+preprint+ option, but
  increases the baselineskip to facilitate easier review process.

\item [{\tt\color{verbcolor} 1p}]  formats the article to the look and feel of the final
  format of model 1+ journals. This is always single column style.

\item [{\tt\color{verbcolor} 3p}] formats the article to the look and feel of the final
  format of model 3+ journals. If the journal is a two column
model, use \verb+twocolumn+ option in combination.

\item [{\tt\color{verbcolor} 5p}] formats for model 5+ journals. This is always
  of two column style.

\item [{\tt\color{verbcolor} authoryear}] author-year citation style of
  \file{natbib.sty}. If you want to add extra options of
\file{natbib.sty}, you may use the options as comma delimited
strings as arguments to \verb+\biboptions+ command. An example
would be:
\end{description}
\begin{vquote}
 \biboptions{longnamesfirst,angle,semicolon}
\end{vquote}

\begin{description}
\item [{\tt\color{verbcolor} number}] numbered citation style. Extra options
  can be loaded with\linebreak \verb+\biboptions+ command.

\item [{\tt\color{verbcolor} sort\&compress}] sorts and compresses the
numbered citations. For example, citation [1,2,3] will become [1--3].

\item [{\tt\color{verbcolor} longtitle}] if front matter is unusually long, use
  this option to split the title page across pages with the correct
placement of title and author footnotes in the first page.

\item [{\tt\color{verbcolor} times}] loads \file{txfonts.sty}, if available in
  the system to use Times and compatible math fonts.

\item[] All options of \file{article.cls} can be used with this
  document class.

\item[] The default options loaded are \verb+a4paper+, \verb+10pt+,
  \verb+oneside+, \verb+onecolumn+ and \verb+preprint+.

\end{description}

\section{Frontmatter}
%\hypertarget{preamble}{}

There are two types of frontmatter coding:
\begin{enumerate}[(1)]
\item each author is
connected to an affiliation with a footnote marker; hence all
authors are grouped together and affiliations follow;
\item authors of same affiliations are grouped together and the
relevant affiliation follows this group. An example coding of the first
type is provided below.
\end{enumerate}

\begin{vquote}
 \title{This is a specimen title\tnoteref{t1,t2}}
 \tnotetext[t1]{This document is a collaborative effort.}
 \tnotetext[t2]{The second title footnote which is a longer
    longer than the first one and with an intention to fill
    in up more than one line while formatting.}
\end{vquote}

\begin{vquote}
 \author[rvt]{C.V.~Radhakrishnan\corref{cor1}\fnref{fn1}}
 \ead{cvr@river-valley.com}

 \author[rvt,focal]{K.~Bazargan\fnref{fn2}}
 \ead{kaveh@river-valley.com}

 \author[els]{S.~Pepping\corref{cor2}\fnref{fn1,fn3}}
 \ead[url]{http://www.elsevier.com}
\end{vquote}

\begin{vquote}
 \cortext[cor1]{Corresponding author}
 \cortext[cor2]{Principal corresponding author}
 \fntext[fn1]{This is the specimen author footnote.}
 \fntext[fn2]{Another author footnote, but a little more 
             longer.}
 \fntext[fn3]{Yet another author footnote. Indeed, you can have
    any number of author footnotes.}

 \address[rvt]{River Valley Technologies, SJP Building,
    Cotton Hills, Trivandrum, Kerala, India 695014}
 \address[focal]{River Valley Technologies, 9, Browns Court,
    Kennford, Exeter, United Kingdom}
 \address[els]{Central Application Management,
    Elsevier, Radarweg 29, 1043 NX\\
    Amsterdam, Netherlands}

\end{vquote}

The output of the above TeX source is given in Clips~\ref{clip1} and
\ref{clip2}. The header portion or title area is given in Clip~\ref{clip1} and
the footer area is given in Clip~\ref{clip2}.

\vspace*{6pt}
\def\rulecolor{blue!70}
\src{Header of the title page.}
\includeclip{1}{132 571 481 690}{els1.pdf}
\def\rulecolor{orange}

%\vspace*{6pt}
\def\rulecolor{blue!70}
\src{Footer of the title page.}
\includeclip{1}{122 129 481 237}{els1.pdf}
\def\rulecolor{orange}
\pagebreak

Most of the commands such as \verb+\title+, \verb+\author+,
\verb+\address+ are self explanatory.  Various components are
linked to each other by a label--reference mechanism; for
instance, title footnote is linked to the title with a footnote
mark generated by referring to the \verb+\label+ string of
the \verb=\tnotetext=.  We have used similar commands
such as \verb=\tnoteref= (to link title note to title);
\verb=\corref= (to link corresponding author text to
corresponding author); \verb=\fnref= (to link footnote text to
the relevant author names).  \TeX{} needs two compilations to
resolve the footnote marks in the preamble part.  
Given below are the syntax of various note marks and note texts.

\begin{vquote}
  \tnoteref{<label(s)>}
  \corref{<label(s)>}
  \fnref{<label(s)>}
  \tnotetext[<label>]{<title note text>}
  \cortext[<label>]{<corresponding author note text>}
  \fntext[<label>]{<author footnote text>}
\end{vquote}

\noindent where \verb=<label(s)>= can be either one or more comma
delimited label strings. The optional arguments to the
\verb=\author= command holds the ref label(s) of the address(es)
to which the author is affiliated while each \verb=\address=
command can have an optional argument of a label. In the same
manner, \verb=\tnotetext=, \verb=\fntext=, \verb=\cortext= will
have optional arguments as their respective labels and note text
as their mandatory argument.

The following example code provides the markup of the second type
of author-affiliation.
%as seen in the output given in the
%box to the right.
%\pinclude{\def\rulecolor{blue!80}
%   \includeclip[width=3.25in]{1}{130 84 484 676}{els2.pdf}%
% \def\rulecolor{orange}}

\begin{vquote}
\author{C.V.~Radhakrishnan\corref{cor1}\fnref{fn1}}
 \ead{cvr@river-valley.com}
 \address{River Valley Technologies, SJP Building,
   Cotton Hills, Trivandrum, Kerala, India 695014}
\end{vquote}

\begin{vquote}
\author{K.~Bazargan\fnref{fn2}}
 \ead{kaveh@river-valley.com}
 \address{River Valley Technologies, 9, Browns Court, Kennford,
   Exeter, UK.}
\end{vquote}

\begin{vquote}
\author{S.~Pepping\fnref{fn1,fn3}}
 \ead[url]{http://www.elsevier.com}
 \address{Central Application Management,
   Elsevier, Radarweg 43, 1043 NX Amsterdam, Netherlands}
\end{vquote}

\begin{vquote}
\cortext[cor1]{Corresponding author}
\fntext[fn1]{This is the first author footnote.}
\fntext[fn2]{Another author footnote, this is a very long 
  footnote and it should be a really long footnote. But this 
  footnote is not yet sufficiently long enough to make two lines 
  of footnote text.}
\fntext[fn3]{Yet another author footnote.}
\end{vquote}

The output of the above TeX source is given in Clip~\ref{clip3}.

\vspace*{12pt}
\def\rulecolor{blue!70}
\src{Header of the title page..}
\includeclip{1}{132 491 481 690}{els2.pdf}
\def\rulecolor{orange}


The frontmatter part has further environments such as abstracts and
keywords.  These can be marked up in the following
manner:

%\verb+\begin{abstract}+ \dots \verb+\end{abstract}+ and
%\verb+\begin{keyword}+ \verb+...+ \verb+\end{keyword}+ which
%contain the abstract and keywords respectively. 

\begin{vquote}
 \begin{abstract}
  In this work we demonstrate the formation of a new type of 
  polariton on the interface between a ....
 \end{abstract}
\end{vquote} 

\begin{vquote}
 \begin{keyword}
  quadruple exiton \sep polariton \sep WGM

  \PACS 71.35.-y \sep 71.35.Lk \sep 71.36.+c
 \end{keyword}
\end{vquote}

\noindent Each keyword shall be separated by a \verb+\sep+ command.
\textsc{pacs} and \textsc{msc} classifications shall be provided in 
the keyword environment with the commands \verb+\PACS+ and
\verb+\MSC+ respectively.  \verb+\MSC+ accepts an optional
argument to accommodate future revisions.
eg., \verb=\MSC[2008]=. The default is 2000.\looseness=-1


\section{Floats}
{Figures} may be included using the command, \verb+\includegraphics+ in
combination with or without its several options to further control
graphic. \verb+\includegraphics+ is provided by \file{graphic[s,x].sty}
which is part of any standard \LaTeX{} distribution.
\file{graphicx.sty} is loaded by default. \LaTeX{} accepts figures in
the postscript format while pdf\LaTeX{} accepts \file{*.pdf},
\file{*.mps} (metapost), \file{*.jpg} and \file{*.png} formats. 
pdf\LaTeX{} does not accept graphic files in the postscript format. 

The \verb+table+ environment is handy for marking up tabular
material. If users want to use \file{multirow.sty},
\file{array.sty}, etc., to fine control/enhance the tables, they
are welcome to load any package of their choice and
\file{elsarticle.cls} will work in combination with all loaded
packages.

\section[Theorem and ...]{Theorem and theorem like environments}

\file{elsarticle.cls} provides a few shortcuts to format theorems and
theorem-like environments with ease. In all commands the options that
are used with the \verb+\newtheorem+ command will work exactly in the same
manner. \file{elsarticle.cls} provides three commands to format theorem or
theorem-like environments: 

\begin{vquote}
 \newtheorem{thm}{Theorem}
 \newtheorem{lem}[thm]{Lemma}
 \newdefinition{rmk}{Remark}
 \newproof{pf}{Proof}
 \newproof{pot}{Proof of Theorem \ref{thm2}}
\end{vquote}

The \verb+\newtheorem+ command formats a
theorem in \LaTeX's default style with italicized font, bold font
for theorem heading and theorem number at the right hand side of the
theorem heading.  It also optionally accepts an argument which
will be printed as an extra heading in parentheses. 

\begin{vquote}
  \begin{thm} 
   For system (8), consensus can be achieved with $\|T_{\omega z}$
   ...
     \begin{eqnarray}\label{10}
     ....
     \end{eqnarray}
  \end{thm}
\end{vquote}  

Clip~\ref{clip4} will show you how some text enclosed between the
above code looks like:

\vspace*{6pt}
\def\rulecolor{blue!70}
\src{{\ttfamily\color{verbcolor}\bs newtheorem}}
\includeclip{2}{1 1 453 120}{jfigs.pdf}
\def\rulecolor{orange}

The \verb+\newdefinition+ command is the same in
all respects as its\linebreak \verb+\newtheorem+ counterpart except that
the font shape is roman instead of italic.  Both
\verb+\newdefinition+ and \verb+\newtheorem+ commands
automatically define counters for the environments defined.

\vspace*{12pt}
\def\rulecolor{blue!70}
\src{{\ttfamily\color{verbcolor}\bs newdefinition}}
\includeclip{1}{1 1 453 105}{jfigs.pdf}
\def\rulecolor{orange}

The \verb+\newproof+ command defines proof environments with
upright font shape.  No counters are defined. 

\vspace*{6pt}
\def\rulecolor{blue!70}
\src{{\ttfamily\color{verbcolor}\bs newproof}}
\includeclip{3}{1 1 453 65}{jfigs.pdf}
\def\rulecolor{orange}

Users can also make use of \verb+amsthm.sty+ which will override
all the default definitions described above.

\section[Enumerated ...]{Enumerated and Itemized Lists}
\file{elsarticle.cls} provides an extended list processing macros
which makes the usage a bit more user friendly than the default
\LaTeX{} list macros.   With an optional argument to the
\verb+\begin{enumerate}+ command, you can change the list counter
type and its attributes.

\begin{vquote}
 \begin{enumerate}[1.]
 \item The enumerate environment starts with an optional
   argument `1.', so that the item counter will be suffixed
   by a period.
 \item You can use `a)' for alphabetical counter and '(i)' for
   roman counter.
  \begin{enumerate}[a)]
    \item Another level of list with alphabetical counter.
    \item One more item before we start another.
    \begin{enumerate}[(i)]
     \item This item has roman numeral counter.
     \item Another one before we close the third level.
    \end{enumerate}
    \item Third item in second level.
  \end{enumerate}
 \item All list items conclude with this step.
 \end{enumerate}
\end{vquote}

\vspace*{12pt}
\def\rulecolor{blue!70}
\src{List -- Enumerate}
\includeclip{4}{1 1 453 185}{jfigs.pdf}
\def\rulecolor{orange}


Further, the enhanced list environment allows one to prefix a
string like `step' to all the item numbers.  Take a look at the
example below:

\begin{vquote}
 \begin{enumerate}[Step 1.]
  \item This is the first step of the example list.
  \item Obviously this is the second step.
  \item The final step to wind up this example.
 \end{enumerate}
\end{vquote}

\def\rulecolor{blue!70}
\src{List -- enhanced}
\includeclip{5}{1 1 313 83}{jfigs.pdf}
\def\rulecolor{orange}

\vspace*{-18pt}

\section{Cross-references}
In electronic publications, articles may be internally
hyperlinked. Hyperlinks are generated from proper
cross-references in the article.  For example, the words
\textcolor{black!80}{Fig.~1} will never be more than simple text,
whereas the proper cross-reference \verb+\ref{tiger}+ may be
turned into a hyperlink to the figure itself:
\textcolor{blue}{Fig.~1}.  In the same way,
the words \textcolor{blue}{Ref.~[1]} will fail to turn into a
hyperlink; the proper cross-reference is \verb+\cite{Knuth96}+.
Cross-referencing is possible in \LaTeX{} for sections,
subsections, formulae, figures, tables, and literature
references.

\section[Mathematical ...]{Mathematical symbols and formulae}

Many physical/mathematical sciences authors require more
mathematical symbols than the few that are provided in standard
\LaTeX. A useful package for additional symbols is the
\file{amssymb} package, developed by the American Mathematical
Society. This package includes such oft-used symbols as
$\lesssim$ (\verb+\lesssim+), $\gtrsim$ (\verb+\gtrsim+)  or 
$\hbar$ (\verb+\hbar+). Note that your \TeX{}
system should have the \file{msam} and \file{msbm} fonts installed. If
you need only a few symbols, such as $\Box$ (\verb+\Box+), you might try the
package \file{latexsym}.

Another point which would require authors' attention is the
breaking up of long equations.  When you use
\file{elsarticle.cls} for formatting your submissions in the 
\verb+preprint+ mode, the document is formatted in single column
style with a text width of 384pt or 5.3in.  When this document is
formatted for final print and if the journal happens to be a double column
journal, the text width will be reduced to 224pt at for 3+
double column and 5+ journals respectively. All the nifty 
fine-tuning in equation breaking done by the author goes to waste in
such cases.  Therefore, authors are requested to check this
problem by typesetting their submissions in final format as well
just to see if their equations are broken at appropriate places,
by changing appropriate options in the document class loading
command, which is explained in section~\ref{sec:usage},
\nameref{sec:usage}. This allows authors to fix any equation breaking
problem before submission for publication.
\file{elsarticle.cls} supports formatting the author submission
in different types of final format.  This is further discussed in
section \ref{sec:final}, \nameref{sec:final}.

\section{Bibliography}

Three bibliographic style files (\verb+*.bst+) are provided ---
\file{elsarticle-num.bst}, \file{elsarticle-num-names.bst} and
\file{elsarticle-harv.bst} --- the first one for the numbered scheme, the
second for the numbered with new options of \file{natbib.sty} and the
last one for the author year scheme.

In \LaTeX{} literature, references are listed in the
\verb+thebibliography+ environment.  Each reference is a
\verb+\bibitem+ and each \verb+\bibitem+ is identified by a label,
by which it can be cited in the text:

\verb+\bibitem[Elson et al.(1996)]{ESG96}+ is cited as
\verb+\citet{ESG96}+. 

\noindent In connection with cross-referencing and
possible future hyperlinking it is not a good idea to collect
more that one literature item in one \verb+\bibitem+.  The
so-called Harvard or author-year style of referencing is enabled
by the \LaTeX{} package \file{natbib}. With this package the
literature can be cited as follows:

\begin{enumerate}[\textbullet]
\item Parenthetical: \verb+\citep{WB96}+ produces (Wettig \& Brown, 1996).
\item Textual: \verb+\citet{ESG96}+ produces Elson et al. (1996).
\item An affix and part of a reference:
\verb+\citep[e.g.][Ch. 2]{Gea97}+ produces (e.g. Governato et
al., 1997, Ch. 2).
\end{enumerate}

In the numbered scheme of citation, \verb+\cite{<label>}+ is used,
since \verb+\citep+ or \verb+\citet+ has no relevance in the numbered
scheme.  \file{natbib} package is loaded by \file{elsarticle} with
\verb+numbers+ as default option.  You can change this to author-year
or harvard scheme by adding option \verb+authoryear+ in the class
loading command.  If you want to use more options of the \file{natbib}
package, you can do so with the \verb+\biboptions+ command, which is
described in the section \ref{sec:usage}, \nameref{sec:usage}.  For
details of various options of the \file{natbib} package, please take a
look at the \file{natbib} documentation, which is part of any standard
\LaTeX{} installation.

\subsection*{Displayed equations and double column journals}

Many Elsevier journals print their text in two columns. Since
the preprint layout uses a larger line width than such columns,
the formulae are too wide for the line width in print. Here is an
example of an equation  (see equation 6) which is perfect in a
single column preprint format:

\bigskip
\setlength\Sep{6pt}
\src{See equation (6)}
\def\rulecolor{blue!70}
\includeclip{4}{134 391 483 584}{els1.pdf}
\def\rulecolor{orange}
                 	
\noindent When this document is typeset for publication in a
model 3+ journal with double columns, the equation will overlap
the second column text matter if the equation is not broken at
the appropriate location.

\vspace*{6pt}
\def\rulecolor{blue!70}
\src{See equation (6) overprints into second column}
\includeclip{3}{61 531 532 734}{els-3pd.pdf}
\def\rulecolor{orange}

\pagebreak

\noindent The typesetter will try to break the equation which
need not necessarily be to the liking of the author or as it
happens, typesetter's break point may be semantically incorrect.
Therefore, authors may check their submissions for the incidence
of such long equations and break the equations at the correct
places so that the final typeset copy will be as they wish.

\section{Final print}\label{sec:final}

The authors can format their submission to the page size and margins
of their preferred journal.  \file{elsarticle} provides four
class options for the same. But it does not mean that using these
options you can emulate the exact page layout of the final print copy. 
\lmrgn=3em
\begin{description}
\item [\texttt{1p}:] $1+$ journals with a text area of
384pt $\times$ 562pt or 13.5cm $\times$ 19.75cm or 5.3in $\times$
7.78in, single column style only.

\item [\texttt{3p}:] $3+$ journals with a text area of 468pt
$\times$ 622pt or 16.45cm $\times$ 21.9cm or 6.5in $\times$
8.6in, single column style.

\item [\texttt{twocolumn}:] should be used along with 3p option if the
journal is $3+$ with the same text area as above, but double column
style. 

\item [\texttt{5p}:] $5+$ with text area of 522pt $\times$
682pt or 18.35cm $\times$ 24cm or 7.22in $\times$ 9.45in,
double column style only.
\end{description}

Following pages have the clippings of different parts of
the title page of different journal models typeset in final
format.

Model $1+$ and $3+$  will have the same look and
feel in the typeset copy when presented in this document. That is
also the case with the double column $3+$ and $5+$ journal article
pages. The only difference will be wider text width of
higher models.  Therefore we will look at the
different portions of a typical single column journal page and
that of a double column article in the final format.

\vspace*{2pc}

\begin{center}
\hypertarget{bsc}{}
\hyperlink{sc}{
{\bf [Specimen single column article -- Click here]}
}

\vspace*{2pc}

\hypertarget{bsc}{}
\hyperlink{dc}{
{\bf [Specimen double column article -- Click here]}
}
\end{center}

\newpage
\vspace*{-2pc}
\src{}\hypertarget{sc}{}
\def\rulecolor{blue!70}
\hyperlink{bsc}{\includeclip{1}{121 81 497 670}{els1.pdf}}
\def\rulecolor{orange}


\newpage

\src{}\hypertarget{dc}{}
\def\rulecolor{blue!70}
\hyperlink{bsc}{\includeclip{1}{55 93 535 738}{els-3pd.pdf}}
\def\rulecolor{orange}

\end{document}
